%%Thank Shuang Hu for contributing
\documentclass{article}
\usepackage[a4paper,left=1cm,right=1cm,top=1cm,bottom=1.4cm]{geometry}
\title{207 S1}
\begin{document}
  \maketitle
	\normalsize
	\section{Formula}
	\textbf{Strain Gauge}:
	\begin{itemize}
		\item $\Delta l$:length
		\item $\Delta a$: cross sectional area
		\item Poisson Ratio: $v = -\frac{e_T}{e_L}$
	\end{itemize}
	$$strain guage:G=\frac{\frac{\Delta R}{R}}{e}$$
	$$strain: e=\frac{\Delta L}{L}$$
	$$\Rightarrow \frac{\Delta R}{R}=G\cdot e$$
	$$R_{new}=R_{origin}(1+G\cdot e)$$
	
	\begin{itemize}
		\item Quater Bridge:$V={out}=\frac{1}{4}V_s\cdot G\cdot e$
		\item Half Bridge:$V={out}=\frac{1}{2}V_s\cdot G\cdot e$
		\item Full Bridge:$V={out}=\frac{1}{1}V_s\cdot G\cdot e$
	\end{itemize}
	
	
	
	\textbf{Gauge Factor} $$ G.e = \frac{\Delta R}{R} $$
	$$\frac{\Delta R}{R} = \frac{\Delta \rho}{\rho} + (1+2k)\frac{\Delta L}{L}$$
  Both side divided by $e$, $G = 1 + 2k + \frac{\Delta\rho}{\rho}\frac{1}{e}$, which is larger than unity.
	
	
	\textbf{RTD(Resistance Temperature Device) In Centigrade degree} 
	$$R_T = R_0(1+a_1T + a_2T^2 + \cdots + a_nT^n)$$
	$$\epsilon(T) = R(T) - R(ideal)$$
	
	
	\textbf{Thermistor(Semiconductors) In K}
	$$R = R_0e^{\frac{1}{\beta}(\frac{1}{T} - \frac{1}{T_0})}$$
	
	
	
	\textbf{Displacement Measurement}
	\begin{itemize}
		\item Resistive Potentiometer:$\displaystyle \frac{V_0}{V_S}=\frac{R_{AC}}{R_{AB}}=AC/AB$
		\item Differential Capacitive transducer:
		\begin{enumerate}
			\item Three plate in parallel and move middle one: $\displaystyle V_0=-V_S\frac{\Delta x}{2d} $(Middle plate move to the right C2). Details in LECTURE 6.
			\item Overlap area changing: $\displaystyle V_0=-V_S\frac{\Delta A}{2A} $(Increase overlap)
		\end{enumerate}
	\end{itemize}
	
	
	\textbf{Hamming Code}
	$$2^r = n+1$$
	$$ r = 3.322log(n+1)$$
	
	\textbf{Accelerometer}
	
	Steady-state sensitivity
	$$ S_0 = \frac{Steady\ state \ voltage}{acceleration} $$ 
	Unit: $\frac{V}{g}$
	
	\textbf{First-order System}
	$$x(t) = x(\infty) + (x(0) - x(\infty))e^{-\frac{t}{\tau}}$$
	
	\textbf{Hamming Distance Between A and B}
	$$A \oplus B$$
	
	
	
	\section{Definition}
	\begin{itemize}
		
		\item \textbf{Sensors}: Detect physical variables and give measurable electrical output.
		\item \textbf{Measurand}: The physical quantity being measured.
		\item \textbf{Transducer}: Device which converts one energy to another.
		\item \textbf{Accuracy}:How close the output reading of the instrument is to the correct value.
		typically: $\pm 1\%$ of all scale.
		\item \textbf{Precision}: Describes an instrument’s degree of freedom from random errors
		\item \textbf{Resolution}: Smallest increment of measurand (increment).
		\item \textbf{Tolerance}: Maximum deviation of a manufactured component from specified value.
		\item \textbf{Span}: Mim and Max value of a quantity that the instrument is designed to measure.
		\item \textbf{Sensitivity}: Sensitivity if a measure of change in output of an instrument for a change in measurement input variable.
		\item \textbf{Rosulution}: Resolution is smallest increment of measurand, which can be measured by instruments.
		
		\item \textbf{Nonlinearity}: Nonlinearity is defined as maximum deviation of any of output readings from the approximate transfer function.
		
		\item \textbf{Hysteresis}: Hysteresis is the deviation of sensor's output at a specified point of input signal, when the input signal is approached from opposite direction, it is expressed as maximum hysteresis.
		
		\item \textbf{MEMS}: Micro-Electro-Mechanical Systems. Micro-components integrated on a single chip, which allows the micro-system to control the system.

	\end{itemize}
	
	\section{Devices}
	\textbf{Inductive transducers (LECTURE 7)}
	\begin{itemize}
		\item Linear Variable Differential Transfoemer(LVDT)
		\item Rotray Variable Differential Transfoemer(RVDT)
	\end{itemize}
	
	\textbf{Optical Transducers (LECTURE 7)}
	
	\begin{itemize}
		\item Incremental: When moving to next sector, only 1 bit changes
		\item absolute: Each sector increments by 1
	\end{itemize}
	\noindent Key points:The advantages and disadvantages for each
	\section{Tao Lu}
	\subsection{Transient Response Analysis(Type-1)}
	\begin{enumerate}
		\item Find the point with proper $\frac{\omega}{\omega_n}$ and $\zeta$
		\item Read $\frac{S}{S_0}$ from the graph(sometimes $20log_{10}(\frac{S}{S_0})$).
		\item Convert from dB, then find S
		\item $V = S \times a$, where a is acceleration
		\item $V_{p-p} = 2\times V$
	\end{enumerate}  
	
		\subsection{Transient Response Analysis(Type-2)}
	\begin{enumerate}
	
	 \item Find $S_0$ by applying $S_0 = \frac{Steady\ State\ Response}{Acceleration}$
	 \item The peak $\frac{S}{S_0}$ is the same as $P = \frac{V_P}{V_S}$
	 \item Find  the value P on the curve.
	 \item You know.
	\end{enumerate}
	
	\subsection{Thermalcouple}
	\begin{enumerate}
		\item Denote the voltage with reference temperature $T_1$ is $V_1$
		\item Find voltage at reference temperature $V_2$
		\item The voltage with reference temperature 0 is $V = V_1 + V_2$
		\item Find the temperature.
	\end{enumerate}  
	
	\subsection{RTD(end point linearity)}
	\begin{enumerate}
		\item Write down $R_T = R_0(1+a_1T + a_2T^2 + \cdots + a_nT^n)$
		\item Find R at $T_{max}$
		\item Calculate the slope $\frac{T_{max}-T_{min}}{R_{max} - R_{min}}$
		\item The slope is sensitivity.
		\item $\epsilon(T) = R(T) - R(ideal)$
		\item $\frac{d\epsilon (T)}{dT}$, find $T_0$
		\item Find $\epsilon (T_0)$
		\item $\epsilon(\% FSD) = \frac{\epsilon (T_0)}{R_{Max} - R_{Min}}$
	\end{enumerate}
	
	\subsection{Strain Gauge}
	\begin{itemize}
		\item \textbf{Temperature Compensator}: Does not measure strain, use cross axis.
		\item \textbf{Amplifier}: $Gain = \frac{R_{Right}}{R_Left}$
	\end{itemize}
	
	\subsection{Resistive Potentiometer}
	\begin{itemize}
		\item \textbf{Error}: Caused by the resistance of potentiometer, or resistance of wires.
	\end{itemize}
	
	\subsection{Fuel Tank}
	Capacitance change when dielectric constant changes. The reading is not affected by the movement because the capacitors are connected in parallel, the total C does not change.
	
	
	\subsection{Parity Check}
	\begin{enumerate}
		\item 
	\end{enumerate}
	
	
	\subsection{Accelerometer}
	Advantages of Servo Accelerometers
	\begin{itemize}
		\item Electronically control damping and spring coefficient, easy to get desired characteristics
		\item Low Hysteresis.
	\end{itemize}
	
	\subsection{Noise Elimination}
	\begin{itemize}
		\item \textbf{Capacitive}: \begin{enumerate}
			\item Differential Amplifier
			\item Connect the inner conductor to outer conductor which is grounded
		\end{enumerate}
		
		\item \textbf{Electromagnatic}: \begin{enumerate}
			\item Twist the conductors(cancel induced voltage).
			\item Physical seperation.
		\end{enumerate}
		
		\item \textbf{Ground Loops}: \begin{enumerate}
			\item Remove one of the ground paths, thus converting the system to a single point ground.
			
			\item Isolate one of the ground paths with an isolation transformer, common mode choke, optical coupler, balanced circuitry, or frequency selective grounding.
		\end{enumerate}
	\end{itemize}
	\subsection{Relative Error}
	\begin{enumerate}
		\item Find the variable X.
		\item Find the error may occur $\Delta X$
		\item Relative error is given by $\frac{\Delta X}{X}$ 
	\end{enumerate}    
	
	\subsection{Phase Dector}
	\begin{enumerate}
		\item Find $V_0 = V_S(\frac{C_1}{C_1 + C_2} - \frac{1}{2})$
		\item Write C as $\epsilon_r\epsilon_0 \frac{A}{d}$, and substitute into the function.
		\item Replace d with $d \pm \Delta X$
		\item The phase indicates the direction
	\end{enumerate}
	
	
\end{document}
\documentclass{article}
\usepackage[a4paper,top=1cm,bottom=2cm,left=1cm,right=1cm]{geometry}
\usepackage{amsmath}
\begin{document}
\Large
  \section{Formulas}
  \textbf{Maxwell's Equations}
\begin{itemize}
\item \textbf{Gauss's Law}  $$ \nabla .D = \rho $$
The divergence of electric flux at a point equals to the charge density at this point.
\item \textbf{Gauss's Law for Magnetic Fields}   $$\nabla .B = 0$$
There're no sink or source of B. B forms close loop.
\item  \textbf{Farady's Law}  $$\nabla \times E = -\frac{\delta B}{\delta t}$$
Electric field E arise due to a time changing magnetic flux density.
\item  \textbf{Ampere's Law}  $$\nabla \times H = J_C + \frac{\delta D}{\delta t}$$
Magnetic field can arise due to conduction current density J or displacement current current density.
\end{itemize}
  \textbf{Electric Field}
  $$E = \frac{Q}{4\pi \epsilon r^2}$$, where $\epsilon = \epsilon_0 \epsilon_r$, $\epsilon_0 = 8.85 \times 10^{-12}Fm^{-1}$
  
  $$E = -grad(V)$$
  
  \textbf{Electric Flux}
  $$\Psi = \iint\epsilon E ds = \iint Dds $$
  
  
  \textbf{Electric Flux Density}
  $$D = \frac{\Psi}{A}$$

  \textbf{Capacitor}
  \begin{itemize}
  \item $ C = \frac{\epsilon A}{d}$
  \item $E = \frac{1}{2}CV^2$
  \end{itemize}      
  
  \textbf{Magnetic Flux}
  $$ \Phi = \iint \mu Hds = \iint Bds $$
  
  \textbf{Magnetic Flux Density}
  $$B = \frac{\Phi}{A} = \mu H$$
  
  
  \textbf{Resistivity}
  $$\rho = \frac{RA}{l}$$
  
  \textbf{Drift Velocity}
  $$U_d = \mu_m E$$
  
  \textbf{Transmission Line}
  \begin{itemize}
  \item \textbf{Shunt Admittance}: $Y = G + j\omega C$
  \item \textbf{Series Impedance}: $Z = R + j \omega L$
  \item \textbf{Propagation Constant}: $ \sqrt{ZY}$
  \item \textbf{Attenuation Constant}:$ Re\sqrt{ZY}$
  \item \textbf{Phase Constant}:$Im\sqrt{ZY} $
  \item \textbf{Characteristic Impedance}:$Z_{line} = \sqrt{\frac{Z}{Y}} = \sqrt{\frac{R + j\omega L}{G + j\omega C}}$
  \item \textbf{Propagation Speed}:$\frac{1}{\sqrt{LC}}$
  \item \textbf{Attenuation(in dB)}: $20log_{10}exp(\alpha \times L)$
  \item \textbf{Voltage Standing-Wave Ratio(VSWR)}: $VSWR = \frac{1+\Gamma_v}{1-\Gamma_v}$
  \item \textbf{Reflection coefficient for V}:$\Gamma_v = \frac{Z_L - Z_0}{Z_L+Z_0}$
  \item \textbf{Reflection coefficient for I}:$\Gamma_i = -\frac{Z_L - Z_0}{Z_L+Z_0} = -\Gamma_v$
  
  \item \textbf{Reflected Voltage}: $V_r = V_i \times \Gamma_v$
  \item \textbf{Reflected Current}: $I_r = I_i \times \Gamma_i$
  \end{itemize}
  
  
  \section{Definitions}
  \begin{itemize}
  \item \textbf{Gauss's Law}: Total electric flux over a volumn is equal to the charge enclosed by that volumn.
  \item \textbf{Electric Field}: E at at a point in a Electric field is the force act on the unit charge at this point.
  \item \textbf{Absolute Potential}: The work move a unit charge from infinity to a radial distance r1. 
  \item \textbf{Electric Flux}: Electric Flux through a surface is the integral of normal component of electric field multiplied by $\epsilon$.
  \item \textbf{Electric Flux Density}: Electric flux divided by A.
  \item \textbf{Permittivity}:Permittivity of vacuum multiplied by relative permittivity.
  \item \textbf{Drift Velocity}: Mobility multiplied by E.
  \item \textbf{Magnatic Flux Density}: B equals to Magnetic flux $\Phi$ divided by area A.
  \item \textbf{Relative Permeability}: Ratio of effective permeability to absolute permeability. 
  \item \textbf{Permeability}:  The degree of magnetization of a material in response to a magnetic field.
  \item \textbf{Transmission Line}: Guide electromagnetic energy or info from one point to another.
  \item \textbf{Application of Transmission Lines}: Telephone, coaxial cables, micro strip tracks on a PCB ......
  \item \textbf{AC Circuit Theory}: $l<<\lambda$
  \item \textbf{Permittivity}: Measures the resistance encountered when forming an electric field.
  \end{itemize}
  
  
  
  \section{Tao Lu}
   \subsection{Know D, find $\rho$}
   \begin{enumerate}
     \item $\iint D ds = \rho$
     \item Determine if the $\rho$ from last step is what we want.
     \item If isn't, for example, we want the $\rho$ of a line, but we have $\rho$ in a volumn, then find the $\rho$ we want.
   \end{enumerate}
   
   \subsection{Magnetic Flux Between Strips}
   \begin{enumerate}
   \item    $ H = \frac{I}{W} $, where W is the width of the strip.
   \item  $\Phi = \mu HA$
   \end{enumerate}
   
   \subsection{Find EMF}
   \begin{enumerate}
   \item Find EMF caused by change of B, $EMF = \frac{d\Phi}{dt} = \frac{A dB}{dt}$
   \item Find EMF caused by $\int (v\times B)dL$
   \item Add them together.
   \end{enumerate}

   \subsection{Wave Equation From Gauss's Law}
   
   \begin{enumerate}
     \item We know $\nabla\times E= -\frac{dB}{dt}$
     \item Calculate curl for both side. $\nabla \times \nabla \times E = \nabla \times \nabla \times -\frac{dB}{dt}$
     \item Substitute $\nabla \times B = \mu_0\epsilon_0\frac{dE}{dt}$ into the equation obtained before
     \item $\nabla \times \nabla \times E  = \nabla (\nabla .E) - \nabla^2E$, where $\nabla .E$ is 0 in vacuum
     \item $\nabla^2E = \mu_0\epsilon_0\frac{dE}{dt}$
   \end{enumerate}
  
\end{document}